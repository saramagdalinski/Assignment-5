\documentclass[12pt]{article}
\usepackage{enumerate}
\usepackage{amsmath}
\usepackage{amssymb}
\usepackage{amsthm}
\usepackage{etoolbox}
\usepackage{graphicx}

\newcommand{\Name}[1]{\noindent \textbf{Name:} #1 \\}
\newcommand{\Workedwith}[1]{\noindent \textbf{Worked with:} #1 \\}
\newcommand{\Problem}[3]{\mbox{} \newline \noindent \textbf{\textbf{Problem #1: #2 [#3 Points] \\ }}}


\begin{document}

\begin{center}
  \bf
  Algorithms \\
  CMPT 307 D200 \\
  Spring 2024 \\
  \rm
  Homework 5\\
  Due:  Sunday, Mar 31 at 10:00 PM \\
\end{center}

\Name{Sara Magdalinski}
\Workedwith{N/A}

\Problem{3}{Fast Flights}{20}

You are working for unAmerican Airlines, and they would like to build a new app called ``Fast Flights'' that offers flyers the fastest possible route from a given starting city to a destination city, with an extra cool feature that is currently not offered on any other service. Fast Flights has the daily schedules of all unAmerican non-stop flights on the planet, where each flight is specified by a starting city, destination city, departure time, and arrival time.  All times are specified in Coordinated Universal Time for uniformity.

A passenger specifies her starting city, destination city, and a rational number greater than or equal to 1 called the \emph{layover penalty.}  For example, if a layover between two flights is 60 minutes and the layover penalty is $1.5$, this means that the layover should be treated as being $60 \times 1.5 = 90$ minutes.  The layover penalty effectively indicates how much a passenger dislikes intermediate hops.  The system returns the door-to-door duration of the shortest route (which includes, of course, the layover penalty for each layover).  That layover penalty is the shiny new feature!

Assume that there are $n$ cities and that there is some constant $C$ such that each city has at most $C$ incoming and $C$ outgoing flights per day.  Assume that this directed graph is represented as an adjacency list where each edge has an associated ordered pair $(\delta, \alpha)$ where $\delta$ is the departure time and $\alpha$ is the arrival time, both specified in Shmorbodian Standard Time.  So $\alpha - \delta$ is the duration of that flight.

Your task is to (1) describe the fastest possible algorithm that you can for solving the problem of finding the fastest route between one given pair of cities and (2) give its asymptotic worst-case running time as a function of $n$ and $C$.  Rather than developing a new algorithm, you should construct a \emph{new graph} that you can then give to an existing graph algorithm.
You should describe how this new graph is constructed, which existing graph algorithm you will use to achieve the best running time, and finally, analyze the running time of the algorithm (don't forget to consider the time required to build the new graph from the existing one).

Hint: Does your algorithm from problem 2 help with this problem?

\textbf{Solution:}\\
Constructing the graph: We are given a graph G(V,E) where V is the vertex set consisting of the cities which is size n and E is the edge set with a size of at most n * c. We will create a new graph called G'(V',E') where for each city (i.e each vertex in V) we will construct a new graph. This will be a bipartite graph G''(V'' = A $\cup$ B, E'') where A is set of incoming flights into that city and B is the set of outgoing flights from that city. Vertices in A and B will be connected by a directed edge from A to B when the outgoing flight from B departs after the incoming flight from A arrives. The weight of this edge will be the calculated layover time * layover penalty (i.e (vertex in B's departure time - vertex in A's arrival time) * layover penalty). These n bipartite graphs will be connected based upon the flights connecting the two cities.  Each outgoing flight from each city will be represented with a directed edge to a vertex in A of another G'' graph. The weight of these edges will be the duration of the specific flight (arrival time - departure time). After cycling through these steps for each city and each flight the overall edge set will be completed.\\

Which algorithm to use: An optimal algorithm to use for this problem is the one constructed for question 2 of this assignment which works for the category of directed acyclic graphs, which this graph falls into. Please see question 2 for a more detailed description of this algorithm. Using this algorithm can provide us with the fastest route from a starting city into a destination city. The distance array will be of size n where n is the number of cities and all values will be set to 0 except for distance[s] = 0 which sets our starting city s to 0. The previous array will be size n as well and will track the flights taken along the fast route, previous[s] will be set to null as we did not take a flight to get to the starting city. For each vertex v in V with V being the vertex set of the topologically sorted graph distance[v] and previous[v] will need to be updated. An incoming flight can be connect to at most C outgoing flights. We can calculate the time by taking the layover time * layover penalty + duration of flight to next city (at which we will repeat this calculation until we have a solution). If the current distance to an adjacent vertex x through v is less than distance[x] we will update distance[x] to be this new distance and we will set previous[x] to u. The distance and previous arrays will be returned and distance[e] will contain the fastest time to get from s to e where e is the end destination. The previous array will provide the flights taken along this route.\\

Time Efficiency: \\

Constructing Graph: We will iterate through all n vertices of the original graph where for each of the n vertices at most 2C vertices are added. This will take O(n * 2C) time since we need to input the times of the flights for these 2c flights. We need to connect the incoming and outgoing flights (to figure out layovers) for each city, since there are at most C incoming flights and at most C outgoing flights this will take O(n * 2C) time. To connect cities together we will consider outgoing flights. Each city has at most C outgoing flights so this takes O(n * C) time. This accounts for incoming flights as well since an outgoing flight from one city will be an incoming flight to another. Therefore, to construct the graph it will take O(n * 2C) + O(n * $C^2$) + O(n * C) time which equals O(n * $C^2$).\\

Running Algorithm from Problem 2: The algorithm we used takes O(m + n) time where m is the number of edges and n is the number of vertices. The graph we created has at most 2NC vertices as for each of the n cities we are adding at most 2C vertices. There will be at most $c^2$ edges within each cities graph as each of the C incoming flights could at most connect to each of the C outgoing flights. There will also be at most C outgoing edges from each city (incoming are considered as they are outgoing flights from another city). This gives us a total of at most n($C^2$ + c) edges = $nC^2 + nC$ edges. So to run the algorithm from problem 2 we will have a time efficiency in the worst case of O(2NC + $nC^2 + nC$) = O(n$C^2$).\\

This brings the overall runtime to solve this problem to O(n$C^2$ + n$C^2$) = O(n$C^2$)

\end{document}
